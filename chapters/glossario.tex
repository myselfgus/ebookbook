%% Glossário

\chapter*{Glossário}
\addcontentsline{toc}{chapter}{Glossário}

\begin{description}[leftmargin=2cm, style=nextline]

\item[Ação Comunicativa]
Conceito de Habermas referente à comunicação orientada ao entendimento mútuo, livre de coerção, visando à emancipação dos interlocutores.

\item[Afeto]
Estado emocional que influencia a cognição e o comportamento. Na perspectiva de Spinoza, os afetos são expressões da potência de existir do ser.

\item[Autonomia]
Capacidade do indivíduo de agir conforme sua própria vontade e valores, livre de determinações externas não reconhecidas criticamente.

\item[Big Five (OCEAN)]
Modelo de personalidade com cinco fatores: Abertura, Conscienciosidade, Extroversão, Amabilidade e Neuroticismo.

\item[Coerência Linguística]
Grau de organização, consistência e fluidez do discurso, refletindo a integração cognitiva e emocional do indivíduo.

\item[Cognição]
Conjunto de processos mentais relacionados ao conhecimento, incluindo atenção, memória, raciocínio e linguagem.

\item[Defusão Cognitiva]
Técnica da ACT que promove o distanciamento dos pensamentos, observando-os sem fusionar-se a eles.

\item[DBT (Terapia Comportamental Dialética)]
Abordagem terapêutica de Marsha Linehan que integra técnicas cognitivo-comportamentais com mindfulness e aceitação.

\item[Emoção]
Estado afetivo complexo que envolve componentes fisiológicos, cognitivos e comportamentais.

\item[Espaço Mental $\mathcal{M}$]
Estrutura matemática em $\mathbb{R}^{15}$ que representa a mente humana através de 15 dimensões fundamentais.

\item[HiTOP]
Hierarchical Taxonomy of Psychopathology --- modelo dimensional de classificação de transtornos mentais.

\item[Linguagem]
Sistema simbólico de comunicação que permite a expressão, organização e transformação do pensamento e da emoção.

\item[Mapa Afetivo]
Representação acumulada das expressões emocionais ao longo do tempo, formando uma trajetória afetiva.

\item[Nomeação]
Ato de dar nome às experiências, conferindo-lhes existência e tornando-as acessíveis à consciência e à comunicação.

\item[PERMA]
Modelo de bem-estar de Seligman: Positive emotions, Engagement, Relationships, Meaning, Accomplishment.

\item[Plasticidade Cerebral]
Capacidade do cérebro de se reorganizar estrutural e funcionalmente em resposta a experiências.

\item[RDoC]
Research Domain Criteria --- framework do NIMH para pesquisa em saúde mental baseado em domínios funcionais.

\item[Reestruturação Cognitiva]
Técnica terapêutica que visa modificar padrões de pensamento disfuncionais.

\item[Self]
Estrutura psicológica que representa a identidade e a subjetividade do indivíduo.

\item[Subjetividade]
Experiência interior e singular do indivíduo, constituída pela interação entre cognição, emoção e linguagem.

\item[Transvaloração]
Conceito nietzschiano referente à reavaliação crítica e criação de novos valores.

\item[WHODAS 2.0]
WHO Disability Assessment Schedule --- instrumento de avaliação de funcionalidade da OMS.

\end{description}

\nextpage
