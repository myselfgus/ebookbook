%% Sobre o Autor

\chapter*{Sobre o Autor}
\addcontentsline{toc}{chapter}{Sobre o Autor}

\textbf{Gustavo Mendes e Silva, M.D.} é médico psiquiatra com interesse particular na interseção entre linguagem, cognição e emoção. Sua abordagem clínica é caracterizada pelo rigor analítico e pela integração de perspectivas filosóficas e científicas.

\section*{Formação Acadêmica}

Graduado em Administração Pública pela UNESP de Araraquara e posteriormente em Medicina pela Faculdade de Medicina de Marília (FAMEMA), com especialização em Psiquiatria. Sua formação multidisciplinar reflete-se em sua abordagem integrativa da mente humana.

\section*{Experiência Profissional}

Durante sua trajetória acadêmica, estagiou na embaixada da ONU no UNFPA (Fundo de População das Nações Unidas) e atuou como diretor nacional de estudantes tanto de Administração Pública quanto de Medicina. Estas experiências proporcionaram-lhe uma perspectiva privilegiada sobre os desafios educacionais e profissionais desses campos, além de uma visão global das questões humanas.

\section*{Interesses de Pesquisa}

Dedica-se ao estudo sistemático da mente humana, combinando observação clínica meticulosa com análise teórica aprofundada. Seu trabalho busca estabelecer pontes entre disciplinas tradicionalmente separadas:
\begin{itemize}
    \item Filosofia da linguagem
    \item Psicologia cognitiva
    \item Neurociência
    \item Matemática aplicada
    \item Teorias da complexidade
\end{itemize}

\section*{Influências Filosóficas}

Seu pensamento é informado por influências filosóficas diversas, desde Bertrand Russell e Wittgenstein até Spinoza e Habermas. Graciliano Ramos e Guimarães Rosa também figuram entre suas referências literárias fundamentais.

\section*{Este Tratado}

O \textit{Tratado Lógico-Afetivo da Linguagem e da Mente Humana} representa a culminação de anos de reflexão sobre a natureza da subjetividade humana e os mecanismos através dos quais a linguagem molda e é moldada pelos processos cognitivos e emocionais.

\vspace{2em}

\begin{center}
\rule{0.3\textwidth}{0.4pt}
\end{center}

\begin{center}
\textit{``As coisas só existem quando nomeadas.''}
\end{center}

\nextpage
