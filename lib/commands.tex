% Comandos e ambientes personalizados para o Tratado Lógico-Afetivo

% Comandos básicos herdados
\newcommand{\latex}{LaTeX\xspace}
\newcommand{\tex}{TeX\xspace}

\def\htbr{\ifdefined\HCode{\HCode{<br/><br/>}}\fi}

\newcommand\nextpage[1][]{
\ifdefined\HCode {
  \HCode{<mbp:pagebreak />}}
\else
  \newpage
\fi
}

\def\smalltt#1{\texttt{\small #1}}
\def\sverb{\Verb[fontsize=\small]}
\def\surl#1{{\small{\url{#1}}}}

% ============================================
% AMBIENTES PARA O TRATADO
% ============================================

% Ambiente para Teses
\newtcolorbox{tese}[1][]{
    enhanced,
    colback=blue!5!white,
    colframe=blue!75!black,
    fonttitle=\bfseries,
    title={\textsc{Tese}},
    #1
}

% Ambiente para Hipóteses
\newtcolorbox{hipotese}[1][]{
    enhanced,
    colback=green!5!white,
    colframe=green!60!black,
    fonttitle=\bfseries,
    title={\textsc{Hipótese}},
    #1
}

% Ambiente para Aforismos
\newtcolorbox{aforismo}[1][]{
    enhanced,
    colback=purple!5!white,
    colframe=purple!75!black,
    fonttitle=\bfseries\itshape,
    title={},
    before upper={\itshape``},
    after upper={''},
    #1
}

% Ambiente para Referências Teóricas
\newtcolorbox{referencia}[1][]{
    enhanced,
    colback=orange!5!white,
    colframe=orange!60!black,
    fonttitle=\bfseries,
    title={\textsc{Referência Teórica}},
    #1
}

% Ambiente para Provas
\newtcolorbox{prova}[1][]{
    enhanced,
    colback=gray!5!white,
    colframe=gray!60!black,
    fonttitle=\bfseries,
    title={\textsc{Prova por Contradição}},
    #1
}

% Ambiente para Axiomas
\newtcolorbox{axioma}[1][]{
    enhanced,
    colback=red!5!white,
    colframe=red!75!black,
    fonttitle=\bfseries,
    title={\textsc{Axioma}},
    #1
}

% Ambiente para Postulados
\newtcolorbox{postulado}[1][]{
    enhanced,
    colback=cyan!5!white,
    colframe=cyan!60!black,
    fonttitle=\bfseries,
    title={\textsc{Postulado}},
    #1
}

% Ambiente para Teoremas
\newtcolorbox{teorema}[1][]{
    enhanced,
    colback=yellow!5!white,
    colframe=yellow!60!black,
    fonttitle=\bfseries,
    title={\textsc{Teorema}},
    #1
}

% Ambiente para Síntese
\newtcolorbox{sintese}[1][]{
    enhanced,
    colback=teal!5!white,
    colframe=teal!60!black,
    fonttitle=\bfseries,
    title={\textsc{Síntese}},
    breakable,
    #1
}

% Ambiente para Epígrafes customizadas
\newcommand{\epigrafe}[2]{%
    \begin{flushright}
    \begin{minipage}{0.7\textwidth}
    \itshape #1
    \begin{flushright}
    --- #2
    \end{flushright}
    \end{minipage}
    \end{flushright}
    \vspace{1em}
}

% Ambiente para diagramas
\newtcolorbox{diagrama}[1][]{
    enhanced,
    colback=white,
    colframe=black!50,
    fonttitle=\bfseries,
    title={\textsc{Diagrama}},
    boxrule=0.5pt,
    #1
}

% ============================================
% ESTILOS TikZ PARA DIAGRAMAS
% ============================================

% Estilos para nós
\tikzstyle{conceito} = [rectangle, rounded corners, minimum width=3cm, minimum height=1cm, text centered, draw=black, fill=blue!20]
\tikzstyle{processo} = [rectangle, minimum width=3cm, minimum height=1cm, text centered, draw=black, fill=orange!20]
\tikzstyle{circulo} = [circle, minimum size=2cm, text centered, draw=black, fill=green!20]
\tikzstyle{triangulo} = [regular polygon, regular polygon sides=3, minimum size=2cm, text centered, draw=black, fill=red!20]

% Estilos para setas
\tikzstyle{seta} = [thick,->,>=stealth]
\tikzstyle{setadupla} = [thick,<->,>=stealth]

% ============================================
% COMANDOS MATEMÁTICOS
% ============================================

% Espaço mental
\newcommand{\espacomental}{\ensuremath{\mathcal{M}}\xspace}
\newcommand{\Rn}[1]{\ensuremath{\mathbb{R}^{#1}}\xspace}

% Operadores
\DeclareMathOperator{\cognicao}{C}
\DeclareMathOperator{\emocao}{E}
\DeclareMathOperator{\linguagem}{L}

% Vetores
\newcommand{\vetor}[1]{\ensuremath{\vec{#1}}}
\newcommand{\vetorneg}[1]{\ensuremath{\boldsymbol{#1}}}

% ============================================
% DEDICATÓRIA
% ============================================

\newenvironment{dedicatoria}{%
    \cleardoublepage
    \thispagestyle{empty}
    \vspace*{\stretch{1}}
    \hfill\begin{minipage}[t]{0.66\textwidth}
    \raggedleft\itshape
}{%
    \end{minipage}
    \vspace*{\stretch{3}}
    \cleardoublepage
}
